\documentclass[a4paper, 12pt]{scrartcl}
% A4 Papier, 12pt Schriftgröße
% scriptartcl, weil article für englischsprachigen Raum optimiert



% --- EINSTELLUNGEN FÜR DEUTSCHE DOKUMENTE:
\usepackage[ngerman]{babel}
% new german (neue deutsche Rechtschreibung)

\usepackage[utf8]{inputenc}
% utf8-Eingabe (Umlaute etc...)

\usepackage[T1]{fontenc}
% damit im erstellten PDF auch nach Wörtern mit Umlauten gesucht werden kann

\usepackage{csquotes}
% für deutsche Anführungsstriche

\title{End-to-end-Learning Ansatz für autonomes Fahren im Miniatur Wunderland}
\author{Nils-Ole Bickel, Michel Brüger}
\date{\today}

\begin{document}
	
\maketitle
% Erstellt den Titel, entsprechend den Angaben in der Präambel
	
\tableofcontents
% Automatisch erstelltes Inhaltsverzeichnis
% zweimal kompilieren, damit Änderungen korrekt angezeigt werden

\newpage
% neue Seite anfangen, damit Titel und Inhaltsverzeichnis auf eigener Seite stehen
	
	\begin{abstract}
ToDo: 

evtl Abstract schreiben	

- end-to-end learning erklären

-- netzt sagt für übertragene Bilder lenkwinkel voraus

-- übermittelte Lenkwinkel als Label, Netz sagt für übertragene Bilder Lenkwinkel voraus
	\end{abstract}

	\section{Hardware}
		\subsection{Raspberry Pi 4}
		Die Hardware in dem Fahrzeug selbst ist nicht stark genug um aus den Kamerabildern den entsprechenden Lenkwinkel zu berechnen. Daher ist das Ziel, die Berechnung der Lenkwinkel auszulagern, auf einen Rechner, der im besten Fall unauffällig im \emph{Miniaturwunderland} "getarnt" werden kann. 
		
		Der \emph{Raspberry Pi 4} hat im Verhältnis zu seiner Größe potente Hardware und verfügt vor allem auch über zwei USB-3-Anschlüsse. Diese sind wichtig um das volle Potential aus dem \emph{Google Coral USB Accelerator} herauszuholen. Außerdem bietet er die Möglichkeit vollwertige Linux-Betriebssysteme zu installieren, was die Verwendung der für die AI-Lenkwinkel-Voraussagung notwendigen Software ermöglicht.
		
		Die Möglichkeit, den \emph{Raspberry Pi} über eine Powerbank per USB-C zu laden, macht es noch leichter ihn in der Kulisse zu verstecken, da man nicht noch ein Kabel zu einem Stromanschluss verlegen muss.
		
		
		\subsection{Google Coral USB Accelerator}
		Der \emph{Google Coral USB Accelerator} ist ein Tensor-Prozessor in Form eines USB-Sticks, der an einen Rechner wie den \emph{Raspberry Pi} angeschlossen werden kann, um die Machine Learning Operationen schnell und besonders energieeffizient durchzuführen. Dies kann der \emph{Raspberry Pi 4} zwar auch alleine bewerkstelligen, der \emph{Google Coral USB Accelerator} ist dabei aber deutlich schneller. Bei spontanen Vergleichen konnten wir eine 14-fache Geschwindigkeit bei der Objekterkennung in Bildern feststellen.
		
		
		\subsection{Das Auto}
		\textbf{\textit{\underline{darüber wissen wir eigentlich nichts, vielleicht Section lieber wieder löschen?}}}
	
	
		\subsection{Der Trainingsrechner}
		Das Netz wurde Trainiert auf einem der Laborrechner in Raum \textbf{\textit{\underline{???}}}. Dieser verfügt über einen \textbf{\textit{\underline{???}}} Prozessor und eine \emph{Nvidia Geforce GTX 1050 Ti} Grafikkarte, welche den Trainingsprozess deutlich beschleunigt, im Vergleich zum Training mit dem Prozessor.	
	
	\section{Software}
		\subsection{Auf dem Raspberry Pi verwendete Software}
			\subsubsection{Python}
			\subsubsection{Tensorflow lite}
		\subsection{Auf dem Auto verwendete Software}
		- irgendwie schickt es einen Videostream...
		\subsection{Auf dem Trainingsrechner verwendete Software}
			\subsubsection{Anaconda}
			- Virtual Environments \\
			- Python \\
			- Tensorflow \\
			- Tensorflow lite converter (heisst der so?) \\
			
		\section{Das Netz}
			\subsection{Implementierung}
			\subsection{Training}
				\subsubsection{Trainingsdaten}
				\subsubsection{Anzahl Epochen}
				\subsubsection{...}
			
	\section{Ergebnisse/Fazit}
	- Genauigkeit der Berechneten Lenkwinkel noch nicht erprobt...
\end{document}