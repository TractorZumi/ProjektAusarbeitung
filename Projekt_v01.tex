\documentclass[a4paper, 12pt]{scrartcl}
% A4 Papier, 12pt Schriftgröße
% scriptartcl, weil article für englischsprachigen Raum optimiert



% --- EINSTELLUNGEN FÜR DEUTSCHE DOKUMENTE:
\usepackage[ngerman]{babel}
% new german (neue deutsche Rechtschreibung)

\usepackage[utf8]{inputenc}
% utf8-Eingabe (Umlaute etc...)

\usepackage[T1]{fontenc}
% damit im erstellten PDF auch nach Wörtern mit Umlauten gesucht werden kann

\usepackage{csquotes}
% für deutsche Anführungsstriche

\title{End-to-end-Learning Ansatz für autonomes Fahren im Miniatur Wunderland}
\author{Nils-Ole Bickel, Michel Brüger}
\date{\today}

\begin{document}
	
\maketitle
% Erstellt den Titel, entsprechend den Angaben in der Präambel
	
\tableofcontents
% Automatisch erstelltes Inhaltsverzeichnis
% zweimal kompilieren, damit Änderungen korrekt angezeigt werden

\newpage
% neue Seite anfangen, damit Titel und Inhaltsverzeichnis auf eigener Seite stehen
	
	\begin{abstract}
ToDo: 

evtl Abstract schreiben	

- end-to-end learning erklären

-- netzt sagt für übertragene Bilder lenkwinkel voraus

-- übermittelte Lenkwinkel als Label, Netz sagt für übertragene Bilder Lenkwinkel voraus
	\end{abstract}

	\section{Hardware}
		\subsection{Raspberry Pi 4}
		\subsection{Google Coral}
		\subsection{Das Auto}
		\subsection{Der Trainingsrechner}
	
	\section{Software}
		\subsection{Auf dem Raspberry Pi verwendete Software}
			\subsubsection{Python}
			\subsubsection{Tensorflow lite}
		\subsection{Auf dem Auto verwendete Software}
		- irgendwie schickt es einen Videostream...
		\subsection{Auf dem Trainingsrechner verwendete Software}
			\subsubsection{Anaconda}
			- Virtual Environments \\
			- Python \\
			- Tensorflow \\
			- Tensorflow lite converter (heisst der so?) \\
			
		\section{Das Netz}
			\subsection{Implementierung}
			\subsection{Training}
				\subsubsection{Trainingsdaten}
				\subsubsection{Anzahl Epochen}
				\subsubsection{...}
			
	\section{Ergebnisse/Fazit}
	- Genauigkeit der Berechneten Lenkwinkel noch nicht erprobt...
\end{document}